\documentclass[% Options of scrbook
  %
  % Preamble fold starts here, unless 'documentclass' is added to the
  % g:vimtex_fold_commands option.
  %
  % draft,
  fontsize=12pt,
  %smallheadings,
  headings=big,
  english,
  paper=a4,
  twoside,
  open=right,
  DIV=14,
  BCOR=20mm,
  headinclude=false,
  footinclude=false,
  mpinclude=false,
  pagesize,
  titlepage,
  parskip=half,
  headsepline,
  chapterprefix=false,
  appendixprefix=Appendix,
  appendixwithprefixline=true,
  bibliography=totoc,
  toc=graduated,
  numbers=noenddot,
]{scrbook}

%
% Fold commands (single)
%
\hypersetup{
  ...,
}
\tikzset{
  testing,
  tested,
}

%
% Fold commands (single_opt)
%
\usepackage[
  ...
]{test}
\usepackage[
  backend=biber,
  style=numeric-comp,
  maxcitenames=99,
  doi=false,
  url=false,
  giveninits=true,
]{biblatex}

%
% Fold commands (multi)
%
\renewcommand{\marginpar}{%
  \marginnote%
}
\newcommand*{\StoreCiteField}[3]{%
  \begingroup
    \global\let\StoreCiteField@Result\relax
    \citefield{#2}[StoreCiteField]{#3}%
  \endgroup
  \let#1\StoreCiteField@Result
}
\newenvironment{theo}[1][]{%
  \stepcounter{theo}%
  \ifstrempty{#1}{%
    \mdfsetup{
      frametitle={%
        \tikz[baseline=(current bounding box.east),outer sep=0pt]%
        \node[anchor=east,rectangle,fill=blue!20] {\strut Theorem~\thetheo};%
      }%
    }%
  }%
  {% else ifstrempty:
    \mdfsetup{
      frametitle={
        \tikz[baseline=(current bounding box.east),outer sep=0pt]%
      \node[anchor=east,rectangle,fill=blue!20]{\strut Theorem~\thetheo:~#1};}%
    }%
  }%
  \mdfsetup{
    innertopmargin=10pt,
    linecolor=blue!20,
    linewidth=2pt,topline=true,
    frametitleaboveskip=\dimexpr-\ht\strutbox\relax,
  }%
  \begin{mdframed}[]\relax%
    }{%
  \end{mdframed}%
}

\begin{document}

Hello World

%
% Fold commands (single)
%
\pgfplotstableread[col sep=semicolon,trim cells]{
x    ; y    ; z   ; type  ; reference       ; comment
0.01 ; 1.00 ; nan ; type1 ; ref-unspecified ;
0.02 ; 2.00 ; nan ; type2 ; ref-unspecified ;
0.03 ; 3.00 ; nan ; type3 ; ref-unspecified ;
}{\datatable}

%
% Fold commands (single)
%
\pgfplotstableread[col sep=semicolon,trim cells]{
x    ; y    ; z   ; type  ; reference       ; comment
0.01 ; 1.00 ; nan ; type1 ; ref-unspecified ;
0.02 ; 2.00 ; nan ; type2 ; ref-unspecified ;
0.03 ; 3.00 ; nan ; type3 ; ref-unspecified ;
}
\datatable

%
% Test for cmd_addplot
%
\begin{tikzpicture}
\begin{axis}
\addplot+[error bars/.cd,x dir=both,x explicit] coordinates {
    (0,0)   +- (0.1,0)
    (0.5,1) +- (0.4,0.2)
    (1,2)
    (2,5)   +- (1,0.1)
};
\end{axis}
\end{tikzpicture}

\section{test 1}

\begin{equation}
  f(x) = 1
  \label{sec:test1}
\end{equation}

\subsection{test 1.1}

\begin{equation}
  f(x) = 1
  \label{sec:test1-longer label}
\end{equation}

\section{test 2}

% {{{ Testing markers
Folded stuff

% }}}

Testing markers %<<:
this fold was NOT recognized by VimTeX before issue #1515

%:>>

%<<:
this fold worked before issue #1515

%:>>

\subsection{test 2.1}

\subsection{test 2.2}

\section{test 3}

\begin{itemize}
  \item test of fold
    folded here
    \begin{enumerate}
      \item new fold level heere
        and here
      \item new fold level heere
        and here
      \item new fold level heere
        and here
      \item and not here
    \end{enumerate}
    folded more here
  \item test of fold
    folded here
\end{itemize}

\begin{enumerate}
  \item \begin{test}
      hello
      world
    \end{test}

  \item Folding issue

  \item New folding issue \begin{proof}
      asdasd
    \end{proof}
\end{enumerate}

% The frames need to be in this order for the test to pass. This is likely a
% bug in vim, see https://github.com/lervag/vimtex/pull/1830#issuecomment-775527621
\begin{frame}
  \frametitle{Title}
  \framesubtitle{Subtitle}

  hello world
\end{frame}

\begin{frame}[noframenumbering]  % Title page
  hello world
\end{frame}

\end{document}
